\documentclass{article}

% TOC
% summary
% description of the sc0
% register map
% instructions
% usage
%
\usepackage{xcolor}
\usepackage{fancyvrb}
\newcommand*\scc[1]{{#1}\textbf{$^\times$}}
\usepackage{geometry}
\geometry{legalpaper, portrait, margin=1in} % formatting for page margins
\usepackage{hyperref}
\hypersetup{
    colorlinks=true,
    linkcolor=black,
    filecolor=red,
    urlcolor=blue,
}

\begin{document}
% document info
\title{\Huge\textbf{Simple Computer 0 Documentation}}
\author{\href{https://github.com/connorl309/}{\textbf{\textcolor{blue}by Connor L.}}}
\date{}
\maketitle

% begin TOC
\begin{tableofcontents}
\pagebreak
% summary of project
\section{\Huge Summary}
\begin{Large}
The SC0 is a hobby project of mine inspired by \href{https://users.ece.utexas.edu/~patt/}{\textbf{Dr. Yale Patt's}}
LC-3(b) teaching computers. In creating the SC0, I am hoping to gain a deeper understanding
of how computer systems are created, and the architecture behind them. I am studying computer
architecture at UT Austin, and am excited to learn more about this field as I progress through
my degree.

The project is written in Rust. I chose this for two reasons: 
\begin{itemize}
    \item Rust has a unique and powerful data flow design
    \item Memory management is completely different (and, in my opinion, stronger) than C++
\end{itemize}
\end{Large}

% description of the sc0 itself
\section{\Huge Description of SC0}
\begin{Large}
The SC0 has several advantages (or, disadvantages, depending on how you look at it) from
the LC-3(b) systems:
\begin{itemize}
    \item 32-bit addressability | more instructions, wider range of values for constants
    \item Multiple types of memory access instructions | direct and indirect
    \item More (useful) instructions | syscalls, an innate stack, logic and math
\end{itemize}
Currently, the SC0 has a fixed address space of 32kB (realistically, no program will ever fill
up this much space, unless I write a C to SC0 compiler). Registers also can store up to 32 bits
of information.

\textit{Note:} A list of system calls and example program formatting can be found later in this
document.
\end{Large}
% register map
\section{\Huge Register Map}
% register map table
\begin{center}
\begin{tabular}{ |c|c|c| }
\hline
\textbf{Register} & \textbf{Type} & \textbf{Notes} \\
\hline
\textbf{R0} & GPR & General purpose \\
\hline
\textbf{R1} & GPR & General purpose \\
\hline
\textbf{R2} & GPR & General purpose \\
\hline
\textbf{R3} & GPR & General purpose \\
\hline
\textbf{R4} & GPR & General purpose \\
\hline
\textbf{R5} & GPR & General purpose \\
\hline
\textbf{R6} & GPR & General purpose \\
\hline
\textbf{R7} & GPR & General purpose \\
\hline
\textbf{R8} & GPR & General purpose \\
\hline
\textbf{R9} & GPR & General purpose \\
\hline
\textbf{R10} & GPR & General purpose \\
\hline
\textbf{R11} & GPR & General purpose \\
\hline
\textbf{R12} & GPR/Return Register & Calls store the return PC in this register \\
\hline
\textbf{R13} & SP & Stack pointer \\
\hline
\textbf{R14} & PC & Current instruction address \\
\hline
\textbf{R15} & PSR & Condition Codes and Privelege \\
\hline
\end{tabular}
\end{center}
\large The special registers \textit{can} be accessed directly, but it is \textbf{highly} advised
to not do that. Condition codes are in the format of N, Z, P, representing negative, zero, and 
positive respectively. Certain instructions set the condition codes (CC) depending on the 
resultant value. Any functions that are user-defined can have any style of parameter passing,
either from the stack or through registers directly.
% instructions
\section{\Huge Instruction Listing and Formatting}
\label{sec:instructionList}
\begin{center}
\begin{tabular}{|c|c|c|}
\hline
\textbf{Instruction} & \textbf{Format} & \textbf{Pseudocode} \\
\hline
\textbf{\scc{ADD}} & add dest, src1, src2/imm & dest = src1 + src2 \\
\hline
\textbf{\scc{SUB}} & sub dest, src1, src2/imm & dest = src1 - src2 \\
\hline
\textbf{\scc{MUL}} & mul dest, src1, src2/imm & dest = src1 * src2 \\
\hline
\textbf{\scc{DIV}} & div dest, src1, src2/imm & dest = src1 / src2, no floats \\
\hline
\textbf{\scc{MOV}} & mov dest, src/imm & dest = src \\
\hline
\textbf{\scc{AND}} & and dest, src1, src2/imm & dest = src1 \& src2 \\
\hline
\textbf{\scc{OR}} & or dest, src1, src2/imm & dest = src1 $|$ src2 \\
\hline
\textbf{\scc{NOT}} & not dest, src & dest = \~{}src \\
\hline
\textbf{\scc{XOR}} & xor dest, src1, src2/imm & dest = src1 \^{} src2 \\
\hline 
\textbf{\scc{LSHF}} & lshf dest, src1, src2/imm & dest = src1 $<$$<$ src2 \\
\hline
\textbf{\scc{RSHF}} & rshf dest, src1, src2/imm & dest = src1 $>$$>$ src2 \\
\hline
\textbf{LEA} & lea dest, LABEL & dest = address of label \\
\hline
\textbf{\scc{LDI}} & ldi dest, src & dest = DWORD(mem[mem[src]]) \\
\hline
\textbf{\scc{LDB}} & ldb dest, src1, src2/imm & dest = BYTE(mem[src1 + src2]) \\
\hline
\textbf{\scc{LDW}} & ldw dest, src1, src2/imm & dest = WORD(mem[src1 + src2]) \\
\hline
\textbf{\scc{LDD}} & ldd dest, src1, src2/imm & dest = DWORD(mem[src1 + src2]) \\
\hline
\textbf{STI} & sti dest, src & DWORD(mem[mem[dest]]) = src \\
\hline
\textbf{STB} & stb dest, src1, src2/imm & BYTE(mem[dest + src2]) = src1 \\
\hline
\textbf{STW} & stw dest, src1, src2/imm & WORD(mem[dest + src2]) = src1 \\
\hline
\textbf{STD} & std dest, src1, src2/imm & DWORD(mem[dest + src2]) = src1 \\
\hline
\textbf{JMP} & jmp reg/LABEL & PC = reg/LABEL address \\
\hline
\textbf{CALL} & call reg/LABEL & R12 = Incremented PC; PC = reg/LABEL address \\
\hline
\textbf{SYSCALL} & syscall CODE & An internal CALL execution \\
\hline
\textbf{BR(nzp)} & BR(nzp) LABEL & Conditional (or unconditional) branch \\
\hline
\textbf{\scc{CMP}} & cmp src1, src2/imm & CC = (src1 - src2) \\
\hline
\textbf{PUSH} & push src/imm & mem[sp++] = src \\
\hline
\textbf{\scc{POP}} & pop dest & dest = mem[$-$$-$sp] \\
\hline
\end{tabular}
\end{center}
\textit{\Large Please note:} instructions marked with $\times$ set condition codes on execution.
% usage
\section{\Huge Usage of SC0}
\begin{Large}
User programs are entered into an SC0 program file that ends in .asm, 
and uses instructions shown above. The typical formatting for an SC0 program
is as follows:
\begin{center}
\textcolor{blue}{\textbf{LABEL:}} \textbf{INSTRUCTION OPERANDS} \textcolor{gray}{ \# comments}
\end{center}
There are four pseudo-ops in the SC0:
\begin{itemize}
    \item \textbf{.ORIG XXXX} | define starting address for program at XXXX
    \item \textbf{.END} | signal end of program for parser
    \item \textbf{.FILL XXXX} | fill that memory location with value XXXX
    \item \textbf{.STRING XXXX} | insert ascii characters in-order, starting at the .STRING memory address
\end{itemize}
The FILL and STRING pseudo-ops can be prefixed with a label, allowing for ease-of-access
in the user program.

Once a user program has been finalized, it may be parsed and assembled by the SC0 program by 
launching the simulator and loading the file into memory.
\end{Large}
% examples and syscalls
\section{\Huge Syscalls and Examples}
\begin{center}
\begin{tabular}{|c|c|}
\hline
\textbf{HALT} & Special call: shuts down SC0. \\
\hline
\textbf{PRINT} & Prints the string found at \textbf{R0} until null terminator. \\
\hline
\textbf{DISPLAY} & Displays data found in R0 on-screen. No automatic newline. \\
\hline
\textbf{DELAY} & Waits \textbf{R0} cycles, then resumes program execution. \\
\hline
\textbf{INPUT} & Reads one character from input into \textbf{R0}. \\
\hline
\end{tabular}
\end{center}
The above system calls are what are implemented in the SC0. Currently, syscalls are executed
on an abstraction layer. In reality, the PC should jump to the syscall value in memory, and read 
the stored address there like a lookup table, then go execute the syscall. The SC0 simply
reads the syscall type and executes it on the Rust layer, not on the assembly layer.

An example program would look as follows:
\begin{center}
    \begin{BVerbatim}
        .ORIG x100 # start program at 0x100
        LEA R0, LOC1
        LEA R1, LOC2
        # load numbers from memory
        LDI R0, R0  
        LDI R1, R1
        ADD R2, R0, R1
        LEA R3, STORE
        STI R3, R2 # store sum of num1 and num2 into mem[0x300]
        SYSCALL HALT
    STORE .FILL 0x300 # store location
    LOC2  .FILL 0x204 # location of number 2
    LOC1  .FILL 0x200 # location of number 1
        .END
    \end{BVerbatim}
\end{center}
As the comments suggest, this program adds the two 4 byte numbers found at \textbf{MEM[0x200]}
and \textbf{MEM[0x204]} and stores the result in \textbf{R2}. Then, the value of \textbf{R2}
is stored into \textbf{MEM[0x300]}. Note the combination of \textcolor{blue}{\textbf{LEA}} and
\textcolor{blue}{\textbf{LDI/STI}}. This is a common combination to use the indirect load and
store instructions. The \textbf{LEA} loads the address of the variables into the specified registers,
and then the \textbf{LDI/STI} perform the chained memory operation as seen in the
\textbf{\hyperref[sec:instructionList]{Instruction Listing}}.
\end{tableofcontents}

\end{document}